\documentclass{article}

\usepackage[letterpaper, portrait, margin=1.5in]{geometry}

\usepackage{fancyhdr}
\usepackage{ragged2e}
\usepackage{graphicx}
\usepackage{caption}
\usepackage{amsmath}
\usepackage{rotating}

\usepackage{listings}
\usepackage{color}

\definecolor{dkgreen}{rgb}{0,0.6,0}
\definecolor{gray}{rgb}{0.5,0.5,0.5}
\definecolor{mauve}{rgb}{0.58,0,0.82}

\lstset{frame=tb,
  language=Java,
  aboveskip=3mm,
  belowskip=3mm,
  showstringspaces=false,
  columns=flexible,
  basicstyle={\small\ttfamily},
  numbers=none,
  numberstyle=\tiny\color{gray},
  keywordstyle=\color{blue},
  commentstyle=\color{dkgreen},
  stringstyle=\color{mauve},
  breaklines=true,
  breakatwhitespace=true,
  tabsize=4
}

\setcounter{secnumdepth}{1}

\usepackage{chngcntr}
\counterwithin{figure}{section}

\renewcommand*{\thepage}{C\arabic{page}}

\pagestyle{fancy}
\lhead{ACME Robotics}
\chead{\#8367}
\rhead{\ifcontents Contents \else Week \thesection \fi}

\newif\ifcontents
\contentstrue

\makeatletter
\renewcommand{\@seccntformat}[1]{}
\makeatother
\begin{document}
\subsection{Planning the Nevada County Student Hackathon}
%! entry
With the Hackathon not to far away, ACME needed to finalize their plans for the event. They resolved the schedule for the day, made certificates for the awards, bought prizes for the raffle, gathered ACME swag to hand out to the kids, checked out extra Chromebooks from the library, and talked to an employee from AJA about how to put on a Hackathon. 

ACME decided to have the Hackathon go from 9 A.M. to 5 P.M. Although this is not a very long time at all by some Hackathon's standards, the team figured this was probably as long as they could go before the kids started to lose interest. They decided to begin the day by having the kids pair up and brainstorm an idea for their game. Then, for the kids who already know Scratch or another language, they would begin to program. For the kids who came with no experience, they would have the choice to attend a Scratch workshop where Emma would teach them how to use Scratch. At noon, they would all have lunch, kindly provided by the Nevada County Tech Connection. Then the kids would keep working on their games until 3:45, where they would begin to prepare for their presentations. From 4 until 5, the kids would give their presentations and then the team would to the raffle and hand out awards. 

The team decided to have three award categories. The Fan Favorite, the Inspiration Award, and the Judges' Award. The Fan Favorite is a chance for the kids to vote for what their favorite game was. The Inspiration award is sort of like the most improved award for the kids that came into the day knowing nothing, and got a lot out of it. The Judges' Award is where all of the ACME members (and mentors) vote on which game they thought was the best. After deciding on these categories, Emma then made certificates for each award. Additionally, they decided to have a raffle. They bought some certificates to movie theater and for a local ice cream and candy parlor. 

Some of the kids mentioned in their registration that they wouldn't be able to bring their own computer to the Hackathon, so to make sure that everyone had a computer, ACME went to the local library and checked out Chromebooks, as Chromebooks can run Scratch on them. 

ACME also invited Garrit from AJA to talk to them about running a Hackathon. Garrit has run adult Hackathons for many years and ACME hoped he could give them some insight. Garrit came to an ACME meeting and talked to members about their schedule for the Hackathon, potentially having a theme for the kids to work off of, how to keep the kids expectations low - so they don't get disappointed when they can't program a first person shooter in Scratch - and overall how to help the kids brainstorm ideas. It was really helpful to talk to Garrit to get some more insight on running and Hackathon and what the team could expect from the day. 

\end{document}