\documentclass{article}

\usepackage{fancyhdr}
\usepackage{ragged2e}
\usepackage{graphicx}
\usepackage{caption}
\usepackage{geometry}
\usepackage{amsmath}
\usepackage{rotating}

\usepackage{listings}
\usepackage{color}

\definecolor{dkgreen}{rgb}{0,0.6,0}
\definecolor{gray}{rgb}{0.5,0.5,0.5}
\definecolor{mauve}{rgb}{0.58,0,0.82}

\lstset{frame=tb,
  language=Java,
  aboveskip=3mm,
  belowskip=3mm,
  showstringspaces=false,
  columns=flexible,
  basicstyle={\small\ttfamily},
  numbers=none,
  numberstyle=\tiny\color{gray},
  keywordstyle=\color{blue},
  commentstyle=\color{dkgreen},
  stringstyle=\color{mauve},
  breaklines=true,
  breakatwhitespace=true,
  tabsize=4
}

\setcounter{secnumdepth}{1}

\usepackage{chngcntr}
\counterwithin{figure}{section}

\renewcommand*{\thepage}{C\arabic{page}}

\pagestyle{fancy}
\lhead{ACME Robotics}
\chead{\#8367}
\rhead{\ifcontents Contents \else Week \thesection \fi}

\newif\ifcontents
\contentsfalse

\makeatletter
\renewcommand{\@seccntformat}[1]{}
\makeatother

\begin{document}

\subsection{Write fundraising letters}
%! Write two fundraising letters, one for existing sponsors and one for new sponsors. 
Since establishing ARES Robotics, the two teams plan to fundraise and go to outreach events together. Since ACME has more experience with writing fundraising letters, Emma was tasked with writing them. She wrote one letter for existing sponsors and then modified it to explain the teams and FIRST for potential sponsors. There are four levels of sponsorship for ACME and ARES. All of them give sponsors benefits, such as their company name/logo on the team website, company name/logo on the side of the robot, and having a private presentation to the sponsoring company. ACME had these support levels last year as well and they worked very well because the sponsor could choose their support level based on what incentives they wanted. The ARES captain, Sean, looked over the letter and approved it. After, the teams developed a potential sponsor list. and wrote another fundraising letter targeted to new sponsors.  

\subsection{Discuss the Engineering Notebook process with both teams}
%! Have an in-depth discussion about the Engineering Notebook process with both ACME and ARES. 
With all of the new team members on ACME (and with ARES being a rookie team) it made sense to discuss the Engineering Notebook as a group. Emma made a presentation on the general setup and format of an Engineering Notebook (EN) and presented it to both teams. Then they split up into teams and Kelly discussed the specifics of the ACME EN with members. He went over how ACME uses Overleaf to format the EN and how this year members will be writing their entries straight into Overleaf. 

The Captains decided to write all of the EN entries in Overleaf because it is more simplistic process than what they did the previous year (i.e. writing entries in Google Sheets and then transferring them into \LaTeX). Since ACME is a smaller team this year, it was fairly simple to teach everyone how to use the platform. Having the entries written directly into Overleaf also allows the EN to be ready for competition earlier because each week can be generated when members finish writing their entries. 

\end{document}