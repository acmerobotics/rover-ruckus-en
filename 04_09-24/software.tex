\documentclass{article}

\usepackage{fancyhdr}
\usepackage{ragged2e}
\usepackage{graphicx}
\usepackage{caption}
\usepackage{geometry}
\usepackage{amsmath}
\usepackage{rotating}

\usepackage{listings}
\usepackage{color}

\definecolor{dkgreen}{rgb}{0,0.6,0}
\definecolor{gray}{rgb}{0.5,0.5,0.5}
\definecolor{mauve}{rgb}{0.58,0,0.82}

\lstset{frame=tb,
  language=Java,
  aboveskip=3mm,
  belowskip=3mm,
  showstringspaces=false,
  columns=flexible,
  basicstyle={\small\ttfamily},
  numbers=none,
  numberstyle=\tiny\color{gray},
  keywordstyle=\color{blue},
  commentstyle=\color{dkgreen},
  stringstyle=\color{mauve},
  breaklines=true,
  breakatwhitespace=true,
  tabsize=4
}

\setcounter{secnumdepth}{1}

\usepackage{chngcntr}
\counterwithin{figure}{section}

\renewcommand*{\thepage}{C\arabic{page}}

\pagestyle{fancy}
\lhead{ACME Robotics}
\chead{\#8367}
\rhead{\ifcontents Contents \else Week \thesection \fi}

\newif\ifcontents
\contentstrue

\makeatletter
\renewcommand{\@seccntformat}[1]{}
\makeatother

\begin{document}\contentsfalse
\subsection{Integrate the hub feedforward functionality into trajectory followers}
%!Integrate the new feedforward functionality into the trajectory code for auto paths.
Before, with each update of the control system, the trajectory would return a target velocity at that given moment, which would be multiplied by the empirically determined coefficient that would convert it to a power to command to the motor. The target position of the trajectory would be compared to the actual position of the robot, and a PID loop would use this error to correct the power commanded to account for the error. Now, the velocity returned by the trajectory is directly commanded to the motor, and the rev hub does the work of converting that to a voltage that is applied to the motor. The positional error still is accounted for, but now by correcting the commanded velocity, not the commanded power. Even with very low gains on the Positional PID loop, this yielded markedly better results than the previous trajectory follower, because of the new functionality in the Rev Hubs. The second necessary change to the trajectory follower was the way error was calculated. Previously, positional error was split into two components, axial and lateral, but the axial error was always parallel to the robot's heading, and the lateral error was always perpendicular to it. Because the new pathing system treated the drive completely holonomically, this no longer made sense, because when the robot was strafing, the axial error would be normal to its direction of travel, but when the robot was driving forward, the axial error would be tangent to its direction of travel. If the robot was moving through a curve with constant heading, always on the path but 1 inch behind where it wanted to be, the error would first be primarily axial, and then primarily lateral, which would cause strange behaviours when these errors are fed into separate PID loops. To counteract this, Kelly changed it so that the axial error was always tangent to the robot's direction of travel, and the lateral error was always normal to the robot's direction of travel. This improved the usefulness of the error measurements.

\begin{lstlisting}[language=Java]
public synchronized Pose2d update(double t, Pose2d pose, TelemetryPacket packet) {
        if (t >= duration) complete = true;
        
        Pose2d targetPose = path.get(axialProfile.get(t).getX());
        double theta = path.deriv(axialProfile.get(t).getX()).pos().angle();
        Pose2d targetVelocity = path.deriv(axialProfile.get(t).getX()).times(axialProfile.get(t).getV());

        Vector2d trackingError = pose.pos().minus(targetPose.pos()).rotated(-theta);
        Vector2d trackingCorrection = new Vector2d(
                axialController.update(trackingError.getX()),
                lateralController.update(trackingError.getY())
        );

        double headingError = pose.getHeading() - targetPose.getHeading();
        double headingCorrection = headingController.update(headingError);

        Pose2d correction = new Pose2d(trackingCorrection, headingCorrection);
        return targetVelocity.plus(correction);
    }
\end{lstlisting}

\end{document}